\documentclass[oneside,final,14pt]{extreport}
\usepackage[utf8]{inputenc}
%\usepackage{G7.32-2017}
\makeatletter
%стиль пишется под класс 'extreport'
%\RequirePackage[14pt]{extsizes}
%\RequirePackage{cmap} % для кодировки шрифтов в pdf
\RequirePackage{mathtext}
\RequirePackage[russian]{babel}
\RequirePackage[utf8]{inputenc}
%\RequirePackage{xltxtra}

%\RequirePackage{polyglossia}
%\setmainlanguage{russian}
%\setotherlanguage{english}
%\setkeys{russian}{babelshorthands=true}

%\setmainfont{Times New Roman}
%\setromanfont{Times New Roman} 
%\setsansfont{Arial} 
%\setmonofont{Courier New} 

%\newfontfamily{\cyrillicfont}{Times New Roman} 
%\newfontfamily{\cyrillicfontrm}{Times New Roman}
%\newfontfamily{\cyrillicfonttt}{Courier New}
%\newfontfamily{\cyrillicfontsf}{Arial}

%\addto\captionsrussian{%
%  \renewcommand{\figurename}{Рисунок}%
% \renewcommand{\tablename}{Таблица}%
%}

\RequirePackage{graphicx} % для вставки картинок
\RequirePackage{amssymb,amsfonts,amsmath,amsthm} % математические дополнения от АМС
\RequirePackage{indentfirst} % отделять первую строку раздела абзацным отступом тоже
\RequirePackage[usenames,dvipsnames]{color} % названия цветов
\RequirePackage{vmargin}
\setpapersize{A4}
\setmarginsrb{3cm}{2cm}{1.5cm}{2cm}{0pt}{0mm}{0pt}{13mm}
\RequirePackage{multirow} % улучшенное форматирование таблиц
\binoppenalty=10000
\relpenalty=10000 % Запрещает перенос формул после знаков бинарных операций и отношений.

\RequirePackage{ulem} % подчеркивания
%\RequirePackage{tikz}
\linespread{1.3} % полуторный интервал
\frenchspacing
\sloppy % Допускает создание максимально разреженных строк для того, чтобы не выйти за поля

%\RequirePackage[datesep={.}]{datetime2} %Дата и время
%\DTMsetstyle{ddmmyyyy}

%\usepackage[backend=biber,style=numeric,sorting=none]{biblatex}
%\usepackage{csquotes}

\newcommand{\RNumb}[1]{\uppercase\expandafter{\romannumeral #1\relax}}

\newcounter{IndAppendix} %Счётчик приложений. Думаю, что желательно его подключить здесь.
\RequirePackage[tableposition=top]{caption} % Красивая подпись к рисункам и графикам
\DeclareCaptionLabelFormat{gostfigure}{Рисунок #2}
\DeclareCaptionLabelFormat{gosttable}{Таблица #2}
\DeclareCaptionLabelSeparator{gost}{~---~}
\captionsetup{labelsep=gost}
\captionsetup[figure]{labelformat=gostfigure}
\captionsetup[table]{labelformat=gosttable}

\newcommand{\@maketablecaption}[2]{ % Подпись к таблице --- слева вверху. Спасибо интернет за это.
\vskip\abovecaptionskip
\sbox\@tempboxa{#1 --- #2}%
\ifdim \wd\@tempboxa >\hsize
#1 --- #2\par
\else
\global \@minipagefalse
\hb@xt@\hsize{\box\@tempboxa}%
\fi
\vskip\belowcaptionskip}
\renewcommand{\table}
{\let\@makecaption\@maketablecaption\@float{table}}

\parindent=1.25cm % Абзацный отступ

\renewcommand\large{\@setfontsize\large{15.5}{17}}

\RequirePackage{enumitem} % Переопределение обычного enumerate и itemize
\AddEnumerateCounter{\arabic}{\@arabic}{}
\setlist{nolistsep}
\renewcommand{\labelitemi}{--- }
\renewcommand{\labelenumi}{\arabic{enumi}}
\renewcommand{\labelenumii}{\arabic{enumi}.\arabic{enumii}.}

% Счётчики общего числа страниц, таблиц, рисунков и источников
\newcounter{totaltable} \setcounter{totaltable} {0}
\newcounter{totalfigure}\setcounter{totalfigure}{0}
\newcounter{totalappendix}

% Команды отображение общего числа ...
\newcommand{\totalpages}{\pageref{reftotalpages}}
\newcommand{\totaltables}{\ref{reftotaltables}}
\newcommand{\totalfigures}{\ref{reftotalfigures}}
\newcommand{\totalappendix}{\ref{reftotalappendix}}

%\RequirePackage[pdfborder={0 0 0}]{hyperref} % Добавляет в документ возможность использовать гиперссылки в PDF
%\RequirePackage{hyperref}


%Переопределение главы:

\renewcommand{\@makechapterhead}[1]{% Начало макроопределения
\vspace*{0 pt}% Пустое место вверху страницы
{
\raggedright \normalfont\large\bfseries
%\@chapapp{} % \@chapapp печатает слово "Глава" (см. ниже)
\hspace{1.25cm} \thechapter \hspace{0.5em}% \par % номер главы - в отдельной строке
%\vspace{20 pt} % между словом "Глава" и ее заголовком
\normalfont\large\bfseries\hyphenpenalty=9999 #1\par % заголовок главы
\nopagebreak % чтоб не оторвать заголовок от текста
\vspace{2 pt} % между заголовком и текстом
}% конец группы.
}% конец макроопределения

\renewcommand{\section}{\@startsection{section}{2}% Секция основного текста
{\parindent}{0pt}{1.5ex}%
{\normalfont\normalsize\bfseries\hyphenpenalty=9999
}}

\renewcommand{\subsection}{\@startsection{subsection}{3}{\parindent}% Подсекция основного текста
{0pt}{1.5ex}%
{\normalfont\normalsize\bfseries\hyphenpenalty=9999
}}

\usepackage{tocloft} %регулировка расположения TableOfContent (Оглавления) на странице

\renewcommand{\cftchapfont}{\normalsize }

\renewcommand{\cfttoctitlefont}{\hspace{0.38\textwidth}\bfseries\MakeUppercase} %уменьшаем размер шрифта и ровняем по центру

% % Межстрочные отступы в Оглавлении:
\setlength{\cftbeforetoctitleskip}{0mm} %отступ Оглавления от верхнего поля страницы.
\setlength{\cftbeforechapskip}{1mm} %отступ между главами
\setlength{\cftbeforesecskip}{1mm} %отступ между секциями \section{title}

% % Отступы от левого поля:
\setlength{\cftchapindent}{0mm} %отступ между левым полем и \chapter{}
\setlength{\cftsecindent}{1.5em} %отступ между левым полем и \section{title}

% % Отточия в Оглавлении
\renewcommand\cftchapdotsep{\cftdot} %добавляет отточия после \chapter{title}
\renewcommand{\cftchapleader}{\cftdotfill{\cftchapdotsep}} %делает отточия после \chapter{title} тонкими, (по умолчанию жирные).
\renewcommand\cftsecdotsep{\cftdot} %делает отточия после \section{title} частыми.
\renewcommand\cftsubsecdotsep{\cftdot}
\cftsetpnumwidth{1.5em} % до какого  момента идут отточия в оглавлении
\setcounter{tocdepth}{2} % задать глубину оглавления — до subsection включительно
\setcounter{secnumdepth}{3}

\AtBeginDocument{\def\contentsname{СОДЕРЖАНИЕ}
\renewcommand\bibname{\vspace{-6ex} \centerline{\large СПИСОК ИСПОЛЬЗОВАННЫХ ИСТОЧНИКОВ}}
%\renewcommand\bibname{{\begin{center} \bfseries\noindent\large СПИСОК ИСПОЛЬЗОВАННЫХ ИСТОЧНИКОВ\end{center}}}
}

%                  Перечисления:

\newenvironment{GostItemize}{%Через тире
\renewcommand{\item}{\par
--- }
}{}

\newcounter{my:enumi}
\newcounter{my:enumii}[my:enumi]
\newcounter{my:enumiii}[my:enumii]

\newenvironment{GostEnumerate}{%Через цифры
\setcounter{my:enumi}{0}
\setcounter{my:enumii}{0}
\setcounter{my:enumiii}{0}
\renewcommand{\item}{\par
\refstepcounter{my:enumi}
\arabic{my:enumi} }

\renewenvironment{GostEnumerate}{
\setcounter{my:enumii}{0}
\setcounter{my:enumiii}{0}
\renewcommand{\item}{\par
\refstepcounter{my:enumii}
\arabic{my:enumi}.\arabic{my:enumii} }

\renewenvironment{GostEnumerate}{
\setcounter{my:enumiii}{0}
\renewcommand{\item}{\par
\refstepcounter{my:enumiii}
\arabic{my:enumi}.\arabic{my:enumii}.\arabic{my:enumiii} }
}{}
}{}
}{}

\newcounter{symbol:enumi} % Символьная нумерация
\newenvironment{GostSymbolEnumerate}{
\setcounter{symbol:enumi}{0}
\renewcommand{\item}{\par
\refstepcounter{symbol:enumi}
\asbuk{symbol:enumi}) }}{}

% Стандартные части

%Титульник
\newcommand{\theauthor}{}
\renewcommand{\author}[1]{\renewcommand{\theauthor}{#1}}
\newcommand{\thegroup}{МТ10-72}
\newcommand{\group}[1]{\renewcommand{\thegroup}{#1}}
\newcommand{\thesubject}{}
\newcommand{\subject}[1]{\renewcommand{\thesubject}{#1}}
\newcommand{\thevariant}{}
\newcommand{\variant}[1]{\renewcommand{\thevariant}{#1}}
\newcommand{\thefaculty}{}
\newcommand{\faculty}[1]{\renewcommand{\thefaculty}{#1}}
\newcommand{\thedepartment}{}
\newcommand{\department}[1]{\renewcommand{\thedepartment}{#1}}
\newcommand{\theteacher}{}
\newcommand{\teacher}[1]{\renewcommand{\theteacher}{#1}}%

\renewcommand{\maketitle}{\newpage\thispagestyle{empty}
\noindent\centerline{\includegraphics[width=145mm]{for_title.png}}
\begin{center}
Московский ордена Ленина, ордена Октябрьской Революции и Ордена Трудового Красного Знамени
государственный технический университет им. Н. Э. Баумана
\end{center}
\vfill
\begin{center}
Факультет <<\thefaculty>>

Кафедра <<\thedepartment>>

\end{center}
\vfill
\begin{center}
Домашнее задание по курсу:%

{\bfseries\large<<\thesubject>>}
\end{center}
\vfill
\centerline{Вариант \thevariant}
\vfill
\begin{flushright}\begin{tabular}{lc}
\multicolumn{2}{c}{Выполнил}\\
Студент & \theauthor\\
\cline{2-2}
Группа & \thegroup\\
\cline{2-2}
 & \\
 Преподаватель & \theteacher\\
 \cline{2-2}
 Дата предъявления & \\
 \cline{2-2}
 & \\
 Дата зачёта & \\
 \cline{2-2}
 Подпись преподавателя & \\
 \cline{2-2}
 \end{tabular} \end{flushright}
\vfill
\vfill
\vfill
\begin{center}
Москва, \today
\end{center}
\clearpage
}


\newcommand{\Executors}{% Список исполнителей
\newpage
\begin{center}
\noindent\bfseries\large СПИСОК ИСПОЛНИТЕЛЕЙ\\
\end{center}}

\newcommand{\Repherat}{% Реферат
\newpage
\begin{center}
\noindent\bfseries\large РЕФЕРАТ\\
\end{center}

%Отчёт \totalpages{} c, \totalfigures{} рис., \totaltables{} табл., \totalappendix{} примеч.,
}

\newcommand{\TermAndDefine}{% Термины и определения
\newpage
\begin{center}
\bfseries\noindent\large ТЕРМИНЫ И ОПРЕДЕЛЕНИЯ\\
\end{center}
\addcontentsline{toc}{chapter}{Термины и определения}
}

\newcommand{\IndividualTask}{% Индивидуальное задание
\newpage
\begin{center}
\bfseries\noindent\large ИНДИВИДУАЛЬНОЕ ЗАДАНИЕ\\
\end{center}
}

\newcommand{\listAbbreviationAndNotation}{% Перечень сокращений и обозначений
\newpage
\begin{center}
\bfseries\noindent\large ПЕРЕЧЕНЬ СОКРАЩЕНИЙ И ОБОЗНАЧЕНИЙ
\end{center}
\addcontentsline{toc}{chapter}{Перечень сокращений и обозначений}
}

\newcommand{\Introduction}{% Введение
\newpage
\begin{center}
\noindent\bfseries\large ВВЕДЕНИЕ
\end{center}
\addcontentsline{toc}{chapter}{Введение}
}

\newcommand{\Conclusion}{%Заключение
\newpage
\begin{center}
\bfseries\noindent\large ЗАКЛЮЧЕНИЕ\\
\end{center}
\addcontentsline{toc}{chapter}{Заключение}
}

\newcommand{\ListSource}[1]{%Список использованных источников
%\begin{center}
%\bfseries\noindent\large СПИСОК ИСПОЛЬЗОВАННЫХ ИСТОЧНИКОВ
%\end{center}

%\bibliographystyle{main_file/utf8gost71}  %% стилевой файл для оформления по ГОСТу
%\bibliography{#1}     %% имя библиографической базы (bib-файла)
%\printbibliography[title={\vspace{-5ex}\begin{center} \bfseries\large СПИСОК ИСПОЛЬЗОВАННЫХ ИСТОЧНИКОВ \end{center}\vspace{-2ex}}]
\addcontentsline{toc}{chapter}{Список использованных источников}
\bibliographystyle{utf8gost705u}  %% стилевой файл для оформления по ГОСТу
\bibliography{#1}     %% имя библиографической базы (bib-файла) 
%\bibliographystyle{utf8gost705u}  %% стилевой файл для оформления по ГОСТу
%\bibliography{#1}
}


\newcommand{\Appendix}{%% Приложение
\newpage\refstepcounter{IndAppendix}
\begin{center}
\bfseries\noindent\large ПРИЛОЖЕНИЕ \Asbuk{IndAppendix}
\end{center}
\addcontentsline{toc}{chapter}{Приложение \Asbuk{IndAppendix}}
%\captionsetup[figure]{labelformat=gostfigure2}
%\captionsetup[table]{labelformat=gosttable2}
\addtocounter{totaltable}{\c@table}
\setcounter{table}{0}
\addtocounter{totalfigure}{\c@figure}
\setcounter{figure}{0}
%\setcounter{section}{\c@IndAppendix}
%\renewcommand{\thesection}{\Asbuk{section}}
\renewcommand{\thefigure}{\Asbuk{IndAppendix}.\arabic{figure}}
\renewcommand{\thetable}{\Asbuk{IndAppendix}.\arabic{table}}
}

%Python в Latex:
\newcommand{\python}[1]{%
\immediate\write18{python #1 > ./py.out 2> ./py.err}
\immediate\input"py.out"}

\AtEndDocument{
\label{reftotalpages}
{\addtocounter{totalfigure}{\c@table}
\addtocounter{totalfigure}{-1}
\refstepcounter{totalfigure}\label{reftotalfigures}}
{\addtocounter{totaltable}{\c@figure}
\addtocounter{totaltable}{-1}
\refstepcounter{totaltable}\label{reftotaltables}}
\addtocounter{totalappendix}{\c@IndAppendix}\label{reftotalappendix}
}

\makeatother

\usepackage{longtable}
\usepackage{lscape}

%\hypersetup{
%	pdfinfo={
%		Title={Домашнее задание по курсу "Теория прокатки"},
%		Author={Солодянкин Андрей},
%		Subject={Технология производства листового и сортового проката},
%		Date={\today},
%		% ...
%	}
%}
\author{Солодянкин А. Д.}
\subject{Технология производства листового и сортового проката}
\teacher{Жихарев П. Ю.}
\variant{12}
\faculty{Машиностроительные технологии}
\department{Оборудование и технологии прокатки}
\begin{document}
%	\maketitle
	\tableofcontents
\chapter{Исходные данные}

Рассматривается участок прокатного стана, на котором осуществляется прокатка полосы в 4 прохода по заданной системе калибровки. Известен материал, температура переднего конца полосы, форма и размеры сечения на входе в первый калибр рассматриваемого участка и размеры сечения после последнего калибра, а также размеры исходной заготовки перед задачей в стан. Задан тип клетей, скорость вращения и диаметр валков, а также длина обводного аппарата линейного стана.

При проведении расчётов принять следующие допущения:

\begin{GostEnumerate}
\item  Расчет температурного режима вести по переднему концу полосы;

\item  Потерями тепла от контакта с валками пренебречь;

\item  Прокатка в непрерывной группе ведётся без натяжения;

\item  В конструкции линейного стана применяются обводные аппараты;

\item  Длина раскатных полей у реверсивной клети не более 30 м с каждой стороны;

\item  Расстояние между клетями в непрерывной групп принять равным 3 м.

\item  Длину пути полосы по обводному аппарату принять равной 4 м.

\end{GostEnumerate}

Требуется:
\begin{GostEnumerate}
\item  рассчитать формоизменение металла при прокатке за 4 прохода: определить коэффициенты вытяжки и обжатия, абсолютные обжатия и степень деформации (результат представить в виде диаграмм);

\item  рассчитать основные размеры калибров и их элементов, достаточные для их построения (результат представить в виде таблицы);

\item  рассчитать скоростной режим прокатки: определить скорости входа и выхода полосы из валков, скорость деформации, длину полосы на всех рассматриваемых участках, машинное время (результат представить в виде таблиц);

\item  рассчитать температурный режим прокатки: определить сопротивление деформации по проходам, а также изменение температуры полосы на всех рассматриваемых участках (результаты представить в виде таблиц или диаграмм);

\item  рассчитать среднее контактное давление, силу и момента прокатки (результаты расчета представить в виде диаграмм). Также в виде таблицы представить результаты расчета коэффициентов, используемых для расчета среднего контактного давления.

\end{GostEnumerate}

Исходная заготовка имеет квадратное сечение. Сторона $ c = 90 $ мм. Длина $ l = 1.5 $ м. Начальное сечение: круг с диаметром $38$ мм. Конечное сечение: квадрат со стороной $C_0=20 $ мм. Схема деформации: овал---круг---овал---диагональный квадрат. Тип стана~--- линейная группа. Материал~--- 12Х18Н9Т. Диаметр валков~--- 300 мм. Скорость вращения валков~--- 230 об/мин. Температура переднего края полосы~--- 1020$^{\circ}C$.

\chapter{Расчёт формоизменения полосы}

Схема деформации представлена на рис. \ref{fig:drawing}. Так как расчёт будет вестись с конца, то принята обратная нумерация калибров. 

\begin{figure}[h!]
	\centering
	\includegraphics[width=16cm,natwidth=610,natheight=642]{pic/drawing.png}
	\caption{Схема деформации полосы}
	\label{fig:drawing}
\end{figure}

\section{Расчёт параметров конечного сечения калибра}

Так как полоса катается при значительной температуре, то необходимо учитывать тепловое расширение металла. Тогда размер конечного профиля с учётом нагрева:
\[ C \approx 1.014 C_0  = 1.014 \cdot 20 = 20.28 \;  {мм} \]
Тогда площадь чистовой заготовки:
\[ F_{ {к}} = C^2 = 411.278 \;  {мм}^2 \]
Размеры калибра расчитываются по формулам, представленным в \cite{mainbook}
\[ H_1' = B_{ {к}} = \sqrt{2} C_1 = \sqrt{2} \cdot 20.28 = 28.68 \;  {мм} \]
\[ r \cong (0.1 \ldots 0.2) \cdot C_1 = (0.1 \ldots 0.2) \cdot 20.28 = 2.028 \ldots 4.056 \approx 3 \;  {мм} \]
\[ H_1 = \sqrt{2} C_1  - 0.83 \cdot r = \sqrt{2} \cdot 20.28 - 0.83 \cdot 3 = 26.190 \;  {мм} \]
\begin{equation}\label{b:pr:kvadrata}
 B_{ {пр}} = B_{ {к}} - S 
\end{equation}
Где $ S $~--- величина межвалкового зазора, определяется согласно рекоммендациям \cite{mainbook}. Учитывая размеры и заданный тип стана:
\[ S = (0.005\ldots0.008) D = (0.005 \ldots 0.008) \cdot 300 = (1.5\ldots2.4) \approx 2 \;  {мм} \]
Тогда из уравнения \ref{b:pr:kvadrata}:
\[ B_{ {вр}} = 28.68 - 2 = 26.68 \;  {мм} \]
\[ r_1 \cong (0.10 \ldots 0.15)H_1' = (0.10 \ldots 0.15) \cdot 28.68 = (2.868 \ldots 4.302) = 3 \;  {мм} \]
\[ \Pi \approx 2 H_1 \sqrt{2} = 2 \cdot 26.190 \cdot \sqrt{2} = 74.077 \;  {мм} \]
Глубина вреза в валки:
\[ H_{ {вр}} = \dfrac{H_1' - s}{2} = \dfrac{28.68 - 2}{2} = 13.34 \;  {мм} \]

\section{Расчёт формоизменения полосы в проходах 1 и 2}

Площадь исходного сечения:
\[ F_{ {исх}} = \dfrac{\pi d_{ {к}}^2}{4} = \dfrac{3.1415 \cdot 38^2}{4} = 1134.082 \;  {мм}^2 \]

Периметр исходного сечения:
\[ \Pi_{ {исх}} = \pi d = 3.1415 \cdot 38 = 119.377 \]

Тогда суммарный коэффициент вытяжки:
\[ \lambda_{\Sigma} = \dfrac{F_{ {исх}}}{F_{ {к}}} =\dfrac{1134.082}{411.278} = 2.757 \]

Средний коэффициент вытяжки за один проход:
\[ \lambda_{ {ср}} = \sqrt[4]{\lambda_{\Sigma}} = \sqrt[4]{2.757} = 1.289 \]

Рассмотрим последние 2 прохода. Прокатка в них осуществляется по схеме круг--овал--диагональный квадрат. Принимая коэффициент вытяжки примерно одинаковым в каждом проходе, определим суммарный коэффициент вытяжки за 2 прохода:
\[ \lambda_{1-2} = \lambda_{ {ср}}^2 = 1.289^2 = 1.662 \]

Приведённый диаметр валков $ A_1: $
\[ A_1 = \dfrac{D - H_1}{H_1} = \dfrac{300 - 26.19}{26.19} = 10.455 \]

Воспользуемся номограммой П15(рис. \ref{fig:nomp15}) в \cite{mainbook} и определим отношение сторон $ a_2 $ овального калибра, которое будет равно $ a_2 = 1.7 $. Далее, воспользовавшись номограммой П13(рис. \ref{fig:nomp13}) определим коэффициент обжатия и фактический коэффициент вытяжки в квадратном калибре. Получим:
\[ \dfrac{1}{\eta_1} = 1.25; \; \lambda_1 = 1.23 \]

\begin{figure}[h!]
	\centering
	\includegraphics[width=16cm,natwidth=610,natheight=642]{pic/NomP15.png}
	\caption{Номограмма П15 для определения суммарного коэффициента вытяжки $ \lambda_{\Sigma} $ при прокатке по схеме круг--овал--квадрат}
	\label{fig:nomp15}
\end{figure}

\begin{figure}[h!]
	\centering
	\includegraphics[width=16cm,natwidth=610,natheight=642]{pic/NomP13.png}
	\caption{Номограмма П13}
	\label{fig:nomp13}
\end{figure}

Тогда коэффициент вытяжки в овальном калибре будет равен:
\[ \lambda_2 = \dfrac{\lambda_{1-2}}{\lambda_1} = \dfrac{1.662}{1.23} = 1.351 \]

Отсюда площадь полосы на входе в чистовой калибр:
\[ F_2 = F_1\cdot \lambda_1 = 411.278 \cdot 1.23 = 505.872 \;  {мм}^2 \]

%Отсюда высота овальной полосы на входе в чистовой калибр составит:
%\[ H_2 = H_1 \cdot \dfrac{1}{\eta} = 26.190 \cdot 1.25 = 32.738 \;  {мм} \]

\[ B_2 = \dfrac{1}{\eta_1} \cdot H_1 = 1.25 \cdot 26.190 = 32.738 \;  {мм} \]

Степень заполнения овального калибра примем $ \delta_2 = 0.85 $, отношение осей $ a_{ {к}2} $ примем равным отношению осей полосы $ a_{ {к}2} = 1.7 $

Тогда, согласно \cite{mainbook} высота овальной полосы составит:
\[ H_2 = \sqrt{\dfrac{F_2}{0.6(2.07 - \delta_2 ) (a_{ {к}} \delta_2 + 0.66 \delta_2 - 0.43)}} \]
\[ H_2 = \sqrt{\dfrac{505.872}{0.6(2.07 - 0.85)(1.7 \cdot 0.85 + 0.66 \cdot 0.85 - 0.43)}} = 20.94 \;  {мм}  \]

Характеристики калибра расчитываются по формулам:
\begin{eqnarray}
R = H_2 \dfrac{(1 + a_{ {к}})^2}{4}; \label{eq:r:oval} \\
B_{ {к}} = H_2 \sqrt{\dfrac{4R}{H_2} - 1}; \label{eq:b:k:oval} \\
B_{ {вр}} = (H_2 - S) \sqrt{\dfrac{4R}{H_2 - S} - 1}; \label{eq:b:vr:oval} \\
r_1 = (0.10\ldots0.40) \cdot H_2; \label{eq:r:1:oval} \\
\Pi \cong 2 \sqrt{B_1^2 + \dfrac{4}{3} H_2^2}; \label{eq:pi:oval} 
\end{eqnarray}

\[ R = 20.94 \cdot \dfrac{\left(1 + 1.7\right)^2}{4} = 38.163 \;  {мм} \]
\[ B_{ {к}} = 20.94 \cdot \sqrt{\dfrac{4 \cdot 38.163}{20.94} - 1} = 52.517 \;  {мм} \]
\[ B_{ {вр}} = (20.94 - 2) \cdot \sqrt{\dfrac{4 \cdot 38.163}{20.94 - 2} - 1} = 50.324 \;  {мм} \]
\[ r_1 = (0.10\ldots0.40) \cdot 20.94 = (2.094\ldots8.376)\approx 5 \;  {мм} \]
\[ \Pi \cong 2 \cdot \sqrt{52.517^2 + \dfrac{4}{3} \cdot 20.94^2} = 115.632 \;  {мм} \]

\section{Расчёт формоизменения полосы в калибрах 3 и 4}

Площадь круглой полосы, выходящей из калибра №3, будет равна:
\[ F_3 = F_1 \cdot \lambda_{1-2} = 411.278 \cdot 1.662 = 683.544 \;  {мм}^2 \]

Примем $ \delta_3 = 0.85 $. Тогда:
\[ H_3 = \sqrt{\dfrac{F_3}{0.785 - 0.667(1 - \delta_3)\sqrt{1 - \delta_3^2}}} = \]
\[ = \sqrt{\dfrac{683.544}{0.785 - 0.667 \cdot(1 - 0.85)\sqrt{1-0.85^2}}}  = 30.552 \;  {мм} \]

Характеристики калибра расчитываются по формулам:
\begin{eqnarray}
B_{ {к}} = \dfrac{D}{\cos\left( \gamma \right)};\label{eq:b:k:krug} \\
B_{ {вр}} = B_{ {к}} - S \tan\left( \gamma \right);\label{eq:b:vr:krug} \\
r_1 (0.08 \ldots 0.10) D;\label{eq:r:1:krug} \\
\Pi = \pi D;\label{eq:pi:krug} \\
H_{ {вр}} = \dfrac{H_1 - S}{2}\label{eq:h:vr:krug}
\end{eqnarray}

Где $ \gamma $~--- угол развала калибра. Так как диаметр круга $ d = 30.552 $, то:
\[ \gamma= 21^{\circ}50' \]
Тогда из уравнений \ref{eq:b:k:krug}, \ref{eq:b:vr:krug}, \ref{eq:r:1:krug}, \ref{eq:pi:krug}, \ref{eq:h:vr:krug}:
\[ B_{ {к}} = \dfrac{30.552}{\cos\left(21^{\circ}50'\right)} = 32.913 \;  {мм} \]
\[ B_{ {вр}} = 32.913 - 2 \cdot \tan\left(21^{\circ}50'\right) = 32.112 \;  {мм} \]
\[ r_1 = (0.08\ldots0.10) = 2.444\ldots3.055 \approx 3 \;  {мм} \]
\[ \Pi = 3.1415 \cdot 30.552 = 95.979 \;  {мм} \]
\[ H_{ {вр}} = \dfrac{30.552 - 2}{2} = 14.276 \;  {мм} \]

Коэффициент обжатия в проходе №2 равен:
\[ \dfrac{1}{\eta_2} = \dfrac{30.552}{20.94} = 1.459 \]

Рассмотрим оставшиеся 2 прохода. Прокатка в них ведётся по схеме круг~---овал~---круг. Суммарный коэффициент вытяжки в проходах № 3 и 4 равен коэффициенту вытяжки в проходах 1 и 2:
\[ \lambda_{3-4} =\lambda_{1-2} = 1.662 \]

Приведённый диаметр валков $ A_3 $:
\[ A_3 = \dfrac{D - H_3}{H_3} = \dfrac{300 - 30.552}{30.552} = 8.819 \]

Воспользуемся номограммой П24(рис. \ref{fig:nomp24}) и определим отношение сторон овального калибра $ a_4 = 1.7 $. Далее, воспользовавшись номограммой П25 (рис. \ref{fig:nomp25}) и определим коэффициент обжатия и коэффициент вытяжки в овальном калибре. Получим:
\[ \lambda_3 = 1.35; \; \dfrac{1}{\eta_3} = 1.48 \]

\begin{figure}[h!]
	\centering
	\includegraphics[width=16cm,natwidth=610,natheight=642]{pic/NomP24.png}
	\caption{Номограмма П24 для определения суммарного коэффициента вытяжки $ \lambda_{\Sigma} $ при прокатке по схеме круг--овал--круг}
	\label{fig:nomp24}
\end{figure}

\begin{figure}[h!]
	\centering
	\includegraphics[width=16cm,natwidth=610,natheight=642]{pic/NomP25.png}
	\caption{Номограмма П25 для определения коэффициента обжатия $ \frac{1}{\eta_1} $ и коэффициента вытяжки $ \lambda_1 $ при прокатке овального раската~--- в круглом калибре}
	\label{fig:nomp25}
\end{figure}

Отсюда коэффициент обжатия в круглом калибре:
\[ \lambda_4 = \dfrac{\lambda_{3-4}}{\lambda_3} = \dfrac{1.662}{1.35} = 1.231 \]

Площадь полосы на выходе из калибра №4:
\[ F_4 = F_3 \cdot \lambda_3 = 683.544 \cdot 1.35 = 922.784  \]

\[ B_4 = H_3 \cdot \dfrac{1}{\eta_3} = 30.552 \cdot 1.48 = 45.217 \;  {мм} \]

Примем степень заполнения калибра $ \delta_4 = 0.85 $, отношение сторон $ a_{ {к} 4} = 1.7 $. Тогда:

Высота полосы на выходе из калибра №4:
\[ H_4 = \sqrt{\dfrac{F_4}{0.6(2.07 - \delta_4 ) (a_{ {к}} \delta_4 + 0.66 \delta_4 - 0.43)}} \]
\[ H_4 = \sqrt{\dfrac{922.784}{0.6(2.07 - 0.85)(1.7 \cdot 0.85 + 0.66 \cdot 0.85 - 0.43)}} = 28.282 \;  {мм}  \]

Коэффициент обжатия в калибре №4:
\[ \dfrac{1}{\eta_4} = \dfrac{d_0}{H_4} = \dfrac{38}{28.282}= 1.343 \]

Параметры калибра расчитываются по формулам \ref{eq:r:oval},  \ref{eq:b:k:oval}, \ref{eq:b:vr:oval}, \ref{eq:r:1:oval}, \ref{eq:pi:oval}:
\[ R = 28.282 \cdot \dfrac{\left(1 + 1.7\right)^2}{4} = 51.544 \;  {мм} \]
\[ B_{ {к}} = 28.282 \cdot \sqrt{\dfrac{4 \cdot 51.544}{28.282} - 1} = 70.939 \;  {мм} \]
\[ B_{ {вр}} = (28.282 - 2) \cdot \sqrt{\dfrac{4 \cdot 51.544}{28.282 - 2} - 1} = 68.760 \;  {мм} \]
\[ r_1 = (0.10\ldots0.40) \cdot 28.282 = 2.828\ldots11.313 = 8 \;  {мм} \]
\[ \Pi \cong 2 \sqrt{70.939^2 + \dfrac{4}{3} \cdot 28.282^2} = 156.190 \;  {мм} \]

\section{Конечные размеры калибров}

Результаты расчёта размеров калибров представлены в табл. \ref{tab:razmer:kalibrov}.

Коэффициенты вытяжки и обжатия по проходам представлены на рис. \ref{fig:lambdaeta}.

\begin{figure}[h!]
	\centering
	\includegraphics[width=0.7\linewidth,natwidth=610,natheight=642]{pic/LambdaEta.png}
	\caption{Значения коэффициентов вытяжки и обжатия по проходам}
	\label{fig:lambdaeta}
\end{figure}


\newpage
\begin{landscape}


%\begin{minipage}{26cm}

\begin{table}[h!]
\caption{Формоизменение полосы и размеры калибров}
\begin{tabular}{|c|c|c|c|c|c|}
	\hline 
	Параметр & Исходное сечение & Калибр №4 & Калибр №3 & Калибр №2 & Калибр №1 \\ 
	\hline 
	Форма прохода & Круг & Овал & Круг & Овал & Диагональный квадрат \\ 
	\hline 
	$H_1$, мм & 38 & 28.282 & 30.552 & 20.94 & 26.190 \\ 
	\hline 
	$H_1'$, мм & - & - & - & - & 28.68 \\ 
	\hline 
	$B_1$, мм & 38 & 45.217 & 30.552 & 32.738 & 26.190 \\ 
	\hline 
	$B_{ {к}}$, мм & - & 70.939 & 32.913 & 52.517 & 28.68 \\ 
	\hline 
	$B_{ {вр}}$, мм & - & 68.760 & 32.112 & 50.324 & 26.68 \\ 
	\hline 
	$r_1$, мм & - & 8 & 3 & 5 & 3 \\ 
	\hline 
	$r$, мм & - & 3 &  & - & 3 \\ 
	\hline 
	$H_{ {вр}}$, мм & - & - & 14.276 & - & 13.34 \\ 
	\hline 
	$R$, мм & - & 51.544 & - & 38.163 & -- \\ 
	\hline 
	$\Pi$, мм & - & 156.190 & 95.979 & 115.632 & 74.077 \\ 
	\hline 
	$\frac{1}{\eta}$ & - & 1.343 & 1.48 & 1.459 & 1.25 \\ 
	\hline 
	$\lambda$ & - & 1.231 & 1.35 & 1.351 & 1.23 \\ 
	\hline 
\end{tabular}
\label{tab:razmer:kalibrov}
\end{table}
%\end{minipage}
\end{landscape}

\chapter{Расчёт энергосиловых параметров}

\section{Формулы для расчёта энергосиловых параметров}

Длина исходной заготовки:
\[ l_0 = 1.5 \cdot \dfrac{90^2}{\pi \dfrac{38^2}{4}} = 10.713 \;  {м} \]

Приведённая высота полосы расчитывается по формуле:
\begin{equation}
\triangle h^{ {пр}} = \dfrac{F_i}{B_i}
\end{equation} 

Абсолютное обжатие расчитывается по формуле:
\begin{equation}\label{eq:abs:obzh}
\triangle h^{ {пр}} = h_0^{ {пр}} - h_1^{ {пр}}
\end{equation}

Относительная деформация расчитывается по формуле:
\begin{equation}\label{eq:otn:def}
\varepsilon = \dfrac{\triangle h^{ {пр}}}{h_0^{ {пр}}}
\end{equation}

Катающий радиус валков расчитываются по формуле:
\[ R_{ {кат} i} =  \dfrac{D - \dfrac{F_i}{B_i}}{2} \]

Длина дуги контакта расчитывается по формуле:
\begin{equation}\label{eq:dlina:dugi}
l = \sqrt{R_{ {кат}} \triangle h^{ {пр}}}
\end{equation}

При прокатке на линейном стане скорость выхода полосы из калибра определяется скоростью вращения валков. Скорость входа полосы определяется коэффициентом вытяжки(опережением и отставанием пренебрегаем):
\[ v_{ {вых}} = \dfrac{\pi D_{ {к}i} n}{60} \]
\[ v_{ {вх} i} = \dfrac{v_{ {вых} i}}{\lambda_i} \]

Скорость деформации расчитывается по формуле:
\begin{equation}\label{eq:u}
u = \dfrac{v \cdot \varepsilon \cdot 1000}{l}
\end{equation}

Сопротивление деформации методом термомеханических коэффициентов:
\begin{equation}
\sigma_{ {ф}} = A_1A_2A_3 \sigma_{ {од}} e^{-m_1t} \varepsilon^{m_2} u^{m3}
\end{equation}

Для стали 12Х18Н9Т:

\[ A_1 A_2 A_3 \sigma_{ {од}} = 3250 \;  {МПа}; \; m_1 = 0.0028; \; m_2 = 0.28; \; m_3 = 0.087 \]

Коэффицент трения по формуле Экелунда с поправками Б. П. Бахтинова и М. М. Штернова:
\begin{equation}
\mu_{ {уст}} = 0.6 k_1 k_2 k_3 (1.05 - 0.0005 T)
\end{equation}

Так как материал валков --- чугун, то \[ k_1 = 0.8 \]
%\[ v > 3 \Rightarrow k_2 = 1.53 v^{-0.467} \]
\[ k_2 = \begin{cases}
1, & v \leqslant 3 \\
1.53 v^{-0.467}, & v > 3
\end{cases} \]
Для стали 12Х18Н9Т \[ k_3 = 1.47 \]

%Для определения уширения воспользуемся формулой Гришкова:
%\begin{equation}
%\triangle b = \dfrac{1}{2} \left( l - \dfrac{\triangle h^{ {пр}}}{2 \mu_{ {уст}}}\right) \ln\left(\dfrac{h_0^{ {пр}}}{h_1^{ {пр}}}\right)
%\end{equation}

Уширение расчитывается по фомуле:
\begin{equation}
\triangle B = B_1 - B_0
\end{equation}

Среднее контактное давление определяется по формуле:
\begin{equation}
p_{ {ср}} = \gamma n_{\sigma}' n_{\sigma}'' n_{\sigma}''' n_{ {в}} n_{ {к}} \sigma_{ {ф}}
\end{equation}

Где:
\[ \gamma = \begin{cases}
1.155 &, \dfrac{b_{ {ср}}}{h_{ {ср}}} > \dfrac{0.465}{\mu_{ {уст}}} \\
1 + \dfrac{\mu_{ {уст}}}{3} \dfrac{b_{ {ср}}}{h_{ {ср}}} &, \dfrac{b_{ {ср}}}{h_{ {ср}}} < \dfrac{0.465}{\mu_{ {уст}}}
\end{cases}
\]

\[n_{\sigma}' = \begin{cases}
\vspace{2ex}
1 + \dfrac{l}{4 h_{ {ср}}} &, \dfrac{l}{h_{ {ср}}} > 4\\
\vspace{2ex}
\dfrac{2 h_{ {н}}}{\triangle h \delta} \left[ \left( \dfrac{h_{ {н}}}{h_{ {1}}^{ {пр}}} \right)^{\delta} - 1\right] &, 2 < \dfrac{l}{h_{ {ср}}} < 4\\
1 + \dfrac{l}{6 h_{ {ср}}}&, 1 < \dfrac{l}{h_{ {ср}}} < 2
\end{cases}
\]
При этом:
\[ \dfrac{h_{ {н}}}{h_1^{ {пр}}} = \left( \dfrac{1 + \sqrt{1 + (\delta^2 - 1) \cdot \left( \dfrac{h_0^{ {пр}}}{h_1^{ {пр}}} \right)^{\delta}}}{\delta + 1} \right)^{\frac{1}{\delta}}
\]

\[ \delta = \dfrac{2 \mu_{ {уст}} l}{\triangle h^{ {пр}}} \]

\[ n_{\sigma}'' = \begin{cases}
1 &, \dfrac{l}{h_{ {ср}}} > 1\\
\left( \dfrac{l}{h_{ {ср}}} \right)^{-0.4} &, \dfrac{l}{h_{ {ср}}} < 1
\end{cases}
\]

\[ n_{\sigma}''' = 1,  {так как натяжение отсутствует} \]

\[ n_{ {в}} = \dfrac{1 + \dfrac{3 b_{ {ср}} - l}{6 b_{ {ср}}} \cdot \mu \frac{l}{h_{ {ср}}}}{1 + \dfrac{\mu}{2} \frac{l}{h_{ {ср}}}}
\]

\[ n_{ {к}} = \dfrac{1.155 \left( 1 + \dfrac{2 \mu_{ {уст}} \alpha l}{3 h_{ {ср}} \pi} \right)}{\gamma \left( 1 + 
	\dfrac{\mu}{3} \dfrac{l}{h_{ {ср}}} \right)} \]

Тогда усилия при прокатке:

\begin{equation}
P = p_{ {ср}} F_{ {к}}
\end{equation}

Где \[ F_{ {к}} = b_{ {ср}} l \]

Момент при прокатке определяется следующим выражением:
\begin{equation}
M = 2 P \psi l
\end{equation}

Где \( \psi \) --- коэффициент плеча, 
\begin{equation}
\psi = \begin{cases}
\vspace{2ex}
\dfrac{1}{2 - \varepsilon} \left( 1 - \varepsilon \left( \dfrac{l^m}{l^m - 1} - \dfrac{1}{m} \right) \right), & m \geqslant 0.5\\

\dfrac{1}{2} \cdot \dfrac{1}{1 - \frac{\varepsilon}{2}} \left( 1 - \varepsilon \cdot \dfrac{1 + m}{2 + m} \right), & m < 0.5
\end{cases}
\end{equation}

При этом:

\begin{equation}
m = \dfrac{\mu_{ {уст}} l}{h_{ {ср}}^{ {пр}}}
\end{equation}

При горячей прокатке в металле практически нет деформационного упрочнения из-за рекристаллизации, поэтому вся работа деформации превращается в тепло.
Температура полосы после клети определяется следующим образом:
\begin{equation}
T_{ {вых}} = T_{ {вх}}^i + \triangle T_{ {д}}^i - \triangle T_{ {в}}^i - \triangle T_{ {л}}^i - \triangle T_{ {к}}^i
\end{equation}

Где \( \triangle T_{ {д}}^i \) --- повышение температуры полосы за счет пластической деформации,
\begin{equation}
\triangle T_{ {д}} = \dfrac{p_{ {ср}} \cdot \ln\left( \dfrac{h_{0}}{h_1} \right) \eta}{c \rho}
\end{equation}

Здесь $c = 548 \dfrac{ {Дж}}{ {кг} \cdot ^{\circ}C} $~--- теплоёмкость стали, $ \rho = 7900 \dfrac{ {кг}}{ {м}^3}$~---
плотность стали при заданной температуре, $\eta = 0.8$~--- потери тепла в окружающую среду.

$\triangle T_{ {в}} $~--- охлаждение полосы за счёт контакта с валками,
\begin{equation}
\triangle T_{ {в}} = \dfrac{4.87}{h_0 + h_1} \left( T_{ {вх}} - T_{ {цв}} \right) \sqrt{\dfrac{2 l h_0}{10^3 \left(h_0 + h_1 \right) v}}
\end{equation}

Здесь $T_{ {цв}}$~--- температура центра валков. Примем $T_{ {цв}} = 50^{\circ}C$.

$\triangle T_{ {л}}$~--- охлаждение полосы за счет лучеиспускания при перемещении ее между клетям,
\begin{equation}
\triangle T_{ {л}} = k_{ {ст}} \dfrac{T_i^4}{h_1^{ {пр}}} \tau_{ {мк}} \cdot 10^{-12}
\end{equation}

Здесь $k_{ {ст}} = 17.5$~--- коэффициент, зависящий от марки сталей, $T_{i} = T_{ {вх}} + \triangle T_{ {д}} - \triangle T_{ {в}}$~---
Температура полосы непосредственно на выходе из валков после $i$--го прохода, $ ^{\circ}K $.

$\tau_{ {мк}}$~--- время прохождения полосы в межклетевом промежутке. 
 
Длина обводного аппарата по условию задания 4 м. После выхода из клети полоса движется до следующей клети по обводному аппарату со скоростью выхода из калибров. Так как скорость входа полосы в следующую клеть меньше, чем скорость входящей полосы, то обводной аппарат раскрывается, и начинает формироваться петля. После раскрытия обводного аппарата оставшаяся часть полосы входит в следующую клеть со скоростью входа, определяемой частотой вращения валков и коэффициентом вытяжки в калибре.
 
Время движения по обводному аппарату составит:
 \[ \tau_{OA} = \dfrac{l_{OA}}{v_{ {вых}}} \]
 
Время нахождения оставшейся части полосы между клетями:
\[ \tau_{ {хв} i} = \dfrac{l_i - l_{OA}}{v_{ {вх} i - 1}} \] 

$\triangle T_{ {к}}$~--- охлаждение полосы за счёт конвекции, 
\begin{equation}
\triangle T_{ {к}} = 0.03 \triangle T_{ {л}}
\end{equation}

Расчёт будем вести в программе tplsp.py. Текст программы приведён в приложениях А, Б.

\section{Исходные данные для расчёта в программе}
\[ F_0 = 1134.082; \; B_0 = 38 \]
\[ F_4 = 922.784; \; B_4' = 45.217; \; B_4 = 28.282 \]
\[ F_3 = 683.544; \; B_3 = 30.552 \]
\[ F_2 = 505.872 \; B_2' = 32.738; \; B_2 = 20.94 \]
\[ F_1 = 411.278 \; B_1 = 26.190 \]
\[ D_{ {в}} = 300 \]
\[ T_0 = 1020 \]
\[ l_{ {мк}} = 4000 \]
\[ n = 230 \]
\[ \lambda_4 = 1.231; \; \lambda_3 = 1.35; \; \lambda_2 = 1.351; \; \lambda_1 = 1.23 \]
\[ l_0 = 10713 \]

\section{Результаты расчёта энергосиловых параметров}

Результаты расчёта представлены в табл. \ref{tab:rez:rasch}.

\begin{table}[h!]
\caption{Результаты расчёта программы tplsp.py}
\begin{tabular}{|c|c|c|c|c|}
\hline
Параметр & Калибр № 4 & Калибр № 3 & Калибр №2 & Калибр №1\\
\hline
$\triangle h$, мм & 9.436 & 10.255 & 6.921 & 8.455\\
\hline
$\varepsilon$ & 0.316 & 0.314 & 0.309 & 0.35\\
\hline
$l$, мм & 36.32 & 37.729 & 31.38 & 34.667\\
\hline
$u$, $\frac{1}{c}$ & 29.311 & 27.851 & 33.78 & 34.562\\
\hline
$\sigma_{ {ф}}$, МПа & 181.616 & 183.925 & 190.953 & 210.884\\
\hline
$\mu_{ {уст}}$ & 0.331 & 0.334 & 0.333 & 0.34\\
\hline
$\triangle b$, мм & 7.217 & 2.27 & 2.186 & 5.25\\
\hline
$b_{ {ср}}$, мм & 41.608 & 29.417 & 31.645 & 23.565\\
\hline
$p_{ {ср}}$, МПа & 231.272 & 226.629 & 244.057 & 259.438\\
\hline
$P_{ {пр}}$, кН & 349.507 & 251.532 & 242.35 & 211.942\\
\hline
$M_{ {пр}}$, кН$\cdot$мм & 8.096 & 8.253 & 10.842 & 13.008\\
\hline
$l_{ {вых}}$, мм & 13187.703 & 17803.399 & 24052.392 & 29584.442\\
\hline
$T_{ {вых}}$, $^{\circ}C$ & 1013.3 & 1004.318 & 981.907 & 964.71\\
\hline
$\triangle T_{ {д}}$, $^{\circ}C$ & 16.243 & 15.801 & 16.692 & 20.65\\
\hline
$\triangle T_{ {в}}$, $^{\circ}C$ & 10.641 & 9.87 & 12.788 & 12.613\\
\hline
$\triangle T_{ {л}}$, $^{\circ}C$ & 11.944 & 14.479 & 25.548 & 24.499\\
\hline
$\triangle T_{ {к}}$, $^{\circ}C$ & 0.358 & 0.434 & 0.766 & 0.735\\
\hline
$\tau$ & 4.898 & 6.639 & 8.372 & 8.641\\
\hline
$v_{ {вх}}$ & 2.735 & 2.477 & 2.536 & 2.783\\
\hline
$v_{ {вых}}$ & 3.367 & 3.343 & 3.427 & 3.424\\
\hline
\end{tabular}
\label{tab:rez:rasch}
\end{table}

Гистограмма абсолютного обжатия полосы представлена на рис. \ref{fig:deltah}.

\begin{figure}[h!]
	\centering
	\includegraphics[width=0.8\linewidth,natwidth=610,natheight=642]{pic/delta_h.png}
	\caption{Абсолютное обжатие полосы по проходам}
	\label{fig:deltah}
\end{figure}

Гистограмма относительной деформации представлена на рис. \ref{fig:epsilon}.

\begin{figure}[h!]
	\centering
	\includegraphics[width=0.8\linewidth,natwidth=610,natheight=642]{pic/epsilon.png}
	\caption{Относительная деформация по проходам}
	\label{fig:epsilon}
\end{figure}

Гистограмма среднего контактного давления по проходам представлена на рис. \ref{fig:psr}.

\begin{figure}[h!]
	\centering
	\includegraphics[width=0.8\linewidth,natwidth=610,natheight=642]{pic/p_sr.png}
	\caption{Среднее контактное давление по проходам}
	\label{fig:psr}
\end{figure}

Гистограмма силы прокатки представлена на рис. \ref{fig:ppr}.

\begin{figure}[h!]
	\centering
	\includegraphics[width=0.8\linewidth,natwidth=610,natheight=642]{pic/P_pr.png}
	\caption{Сила прокатки по проходам}
	\label{fig:ppr}
\end{figure}

Гистограмма момента прокатки представлена на рис. \ref{fig:m}. 

\begin{figure}[h!]
	\centering
	\includegraphics[width=0.8\linewidth,natwidth=610,natheight=642]{pic/M.png}
	\caption{Момент прокатки по проходам}
	\label{fig:m}
\end{figure}

\cite{*}

\ListSource{bibliogr}

%\printbibliography[title={\centerline{\large СПИСОК ИСПОЛЬЗОВАННЫХ ИСТОЧНИКОВ} \addcontentsline{toc}{chapter}{Список использованных источников}}]

\Appendix
\begin{center}
\bfseries\large Текст программы tplsp.py
\end{center}
\label{appendix:tplsp}

\begin{verbatim}
1. from rolling_stand import Stand, LetStripStand
2. 
3. F_4, F_3, F_2, F_1, F_0 = 922.784, 683.544, 505.872,
411.278, 1134.082
4. B_4, B_4_sht, B_3, B_2, B_2_sht, B_1, B_0 = 28.282, 
45.217, 30.552, 32.738, 20.94, 26.19, 38
5. D_val, T_0, l_mk, n, = 300, 1020, 4000, 230
6. lambda_4, lambda_3, lambda_2, lambda_1 = 1.231, 1.35
, 1.351,1.23
7. l_0 = 10_713
8. #-----------вычисляем скорости прокатки-------------
---
9. caliber_4 = LetStripStand(F_0, F_4, B_0, B_4_sht, D_
val, T_0, l_mk, n, lambda_4, l_0, 1)
10. caliber_3 = LetStripStand(F_4, F_3, B_4, B_3, D_val
, T_0, l_mk, n, lambda_3, l_0, 1)
11. caliber_2 = LetStripStand(F_3, F_2, B_3, B_2, D_val
, T_0, l_mk, n, lambda_2, l_0, 1)
12. caliber_1 = LetStripStand(F_2, F_1, B_2_sht, B_1, D
_val, T_0, l_mk, n, lambda_1, l_0, 1)
13. #caliber_4.computing()
14. #caliber_3.computing()
15. #caliber_2.computing()
16. #caliber_1.computing()
17. 
18. #----------вычисляем параметры прокатки------------
----------
19. caliber_4 = LetStripStand(F_0, F_4, B_0, B_4_sht, D
_val, T_0, l_mk, n, lambda_4, l_0, caliber_3.v_vh)

20. caliber_4.computing()
21. caliber_3 = LetStripStand(F_4, F_3, B_4, B_3, D_val
, caliber_4.params[r'$T_{ {вых}}$, $^{\circ}C$'], l
_mk, n, lambda_3, caliber_4.params[r'$l_{ {вых}}$, 
мм'], caliber_2.v_vh)
22. caliber_3.computing()
23. caliber_2 = LetStripStand(F_3, F_2, B_3, B_2, D_val
, caliber_3.params[r'$T_{ {вых}}$, $^{\circ}C$'], l
_mk, n, lambda_2, caliber_3.params[r'$l_{ {вых}}$, 
мм'], caliber_1.v_vh)
24. caliber_2.computing()
25. caliber_1 = LetStripStand(F_2, F_1, B_2_sht, B_1, D
_val, caliber_2.params[r'$T_{ {вых}}$, $^{\circ}C$'
], l_mk, n, lambda_1, caliber_2.params[r'$l_{ {вых}
}$, мм'], caliber_1.v)
26. caliber_1.computing()
27. 
28. ans = []
29. for key in caliber_4.params.keys():
30.     tt = [key, round(caliber_4.params[key], 3), rou
nd(caliber_3.params[key], 3), round(caliber_2.params[ke
y], 3), round(caliber_1.params[key], 3)]
31.     ans.append(tt)
32. 
33. ans = [['Параметр', 'Калибр № 4', 'Калибр № 3', 'Ка
либр №2', 'Калибр №1']] + ans
34. for i in ans:
35.     print(*i, sep=' & ', end='\\\\\n\\hline\n')
36.     
37. 

\end{verbatim}

\Appendix
\begin{center}
	\bfseries\large Текст модуля rolling\_stand
\end{center}

\begin{verbatim}
1. import sympy
2. 
3. class Stand:
4.     def __init__(self, r_kat, h_0, h_1, b_0, t_vh, a
, v, lambd, l_0, v_vh_next = None, b_1 = None):
5.         self.r_kat = r_kat
6.         self.h_0 = h_0
7.         self.h_1 = h_1
8.         self.b_0 = b_0
9.         self.b_1 = b_1
10.         self.t_vh = t_vh
11.         self.a = a
12.         self.v = v
13.         self.lambd = lambd
14.         self.l_0 = l_0
15.         self.params = {}
16.         self.v_vh_next = v_vh_next
17.         self.is_computing = False
18.     
19.     def computing(self):
20.         if self.is_computing:
21.             return None
22.         params = dict()
23.         params[r'$\triangle h$, мм'] = self.h_0 - s
elf.h_1
24.         params[r'$\varepsilon$'] = params[r'$\trian
gle h$, мм'] / self.h_0
25.         params[r'$l$, мм'] = (self.r_kat * params[r
'$\triangle h$, мм']) ** 0.5
26.         params['$u$, $\\frac{1}{c}$'] = self.v * pa
rams[r'$\varepsilon$'] * 1000 / params['$l$, мм']
27.         #params['u'] = 0.105 * n_валков * (params[r
'\varepsilon'] * R_рабочий_валков / self.h_0) ** 0.5
28.         params[r'$\sigma_{ {ф}}$, МПа'] = 3250 
* sympy.exp(-0.0028 * self.t_vh).n(5) * params[r'$\vare
psilon$'] ** (0.28) * params['$u$, $\\frac{1}{c}$'] ** 
(0.087)
29.         def k_2(v):
30.             if v < 3:
31.                 return 1
32.             else:
33.                 return 1.53 * v ** (-0.467)
34.         params[r'$\mu_{ {уст}}$'] = 0.6 * 0.8 *
k_2(self.v) * 1.47 * (1.05 - 0.0005 * self.t_vh)
35.         if self.b_1 is None:
36.             print('b_1 is None!')
37.             params[r'$\triangle b$'] = 1 / 2 * (l -
params[r'$\triangle h$'] / ( 2 * params[r'$\mu_{ {
уст}}$'])) * sympy.ln(self.h_0 / self.h_1).n(5)
38.         else:
39.             params[r'$\triangle b$, мм'] = ((self.b
_1 - self.b_0) ** 2) ** 0.5
40.         params[r'$b_{ {ср}}$, мм'] = self.b_0 +
params[r'$\triangle b$, мм'] / 2
41.         self.b_1 = self.b_0 + params[r'$\triangle b
$, мм']
42.         
43.         def gamma(b_1, b_0, h_1, h_0, mu):
44.             if (((b_1 + b_0) / 2) / ((h_0 + h_1) / 
2) > 0.465 / mu):
45.                 return 1.155
46.             else:
47.                 return 1 + mu / 3 * (((b_1 + b_0) /
2) / ((h_0 + h_1) / 2))
48.         def delta(mu, l, delta_h):
49.             return (2 * mu * l) / (delta_h)
50. 
51.         def h_n(h_1, h_0, delta):
52.             return (((1 + (1 + (delta ** 2 - 1) * (
(h_0 / h_1) ** delta)) ** 0.5) / (delta + 1)) ** (1 / d
elta)) * h_1
53. 
54. 
55.         def n_1(l, h_1, h_0, mu):
56.             if l / ((h_1 + h_0) / 2) > 4:
57.                 return 1 + l / (4 * ((h_1 + h_0) / 
2))
58.             elif l / ((h_1 + h_0) / 2) > 2:
59.                 d = delta(mu, l, (h_0 - h_1))
60.                 hn = h_n(h_1, h_0, d)
61.                 return (2 * hn) / ((h_0 - h_1) * (d
- 1)) * ((hn / h_1) ** d - 1)
62.             else:
63.                 return 1 + l / (6 * ((h_1 + h_0) / 
2))
64. 
65.         def n_2(l, h_1, h_0):
66.             if l / ((h_1 + h_0) / 2) > 1:
67.                 return 1
68.             else:
69.                 return (l / ((h_1 + h_0) / 2)) ** (
-0.4)
70. 
71.         def n_v(l, b_1, b_0, h_1, h_0, mu):
72.             return (1 + ((3 * ((b_1 + b_0) / 2) - l
) / (6 * ((b_1 + b_0) / 2))) * mu * l / ((h_1 + h_0) / 
2)) / (1 + mu / 2 * l / ((h_0 + h_1) / 2))
73. 
74.         def n_k(mu, l, h_1, h_0, gamma):
75.             h_sr = (h_1 + h_0) / 2
76.             return (1.155 * (1 + 2 * mu * 0.9 * l /
(3 * h_sr * sympy.pi.n(5)))) / (gamma * (1 + mu / 3 * 
l / h_sr))
77.         def p_sr(h_1, h_0, b_1, b_0, l, mu, sigma_f
):
78.             gm = gamma(b_1, b_0, h_1, h_0, mu)
79.             n1 = n_1(l, h_1, h_0, mu)
80.             n2 = n_2(l, h_1, h_0)
81.             nv = n_v(l, b_1, b_0, h_1, h_0, mu)
82.             nk = n_k(mu, l, h_1, h_0, gm)
83.             return gm * n1 * n2 * 1 * nv * nk * sig
ma_f
84.         
85.         def k_2(v):
86.             return 1.53 * v ** (-0.467) if v > 3 el
se 1
87. 
88.         def mu(v, t):
89.             return 0.6 * 0.8 * k_2(v) * 1.47 * (1.0
5 - 0.0005 * t)
90.         
91.         p_mid = p_sr(self.h_1, self.h_0, self.b_1, 
self.b_0, params['$l$, мм'], params[r'$\mu_{ {уст}}
$'], params[r'$\sigma_{ {ф}}$, МПа'])
92.         params[r'$p_{ {ср}}$, МПа'] = p_mid
93.         f_k = p_mid * (self.b_1 + self.b_0) / 2 * p
arams['$l$, мм'] # Это в ньютонах
94.         params[r'$P_{ {пр}}$, кН'] = f_k / 1000
# это в килоньютонах
95.         def var_m(mu, l, h_1, h_0):
96.             h_sr = (h_1 + h_0) / 2
97.             return mu * l / h_sr
98. 
99.         def var_psi(epsilon, mu, l, h_1, h_0):
100.             m = var_m(mu, l, h_1, h_0)
101.             if m >= 0.5:
102.                 return 1 / (2 - epsilon) * (1 - ep
silon * (l ** m / (l ** m - 1) - 1 / m))
103.             else:
104.                 return 0.5 * (1 / (1 - epsilon / 2
)) * (1 - epsilon * (1 + m) / (2 + m))
105. 
106.         def mom_prok(epsilon, mu, l, h_1, h_0, p):
107.             psi = var_psi(epsilon, mu, l, h_1, h_0
)
108.             return 2 * p * psi * l / 1000
109.         
110.         mp = mom_prok(params[r'$\varepsilon$'], pa
rams[r'$\mu_{ {уст}}$'], params['$l$, мм'], self.h_
1, self.h_0, p_mid)
111.         params['$M_{\ {пр}}$, кН$\\cdot$мм'] =
mp
112.         
113.         v_vih = self.v * self.lambd
114.         #params[r'v_{ {вых}}'] = v_vih
115.         params[r'$l_{ {вых}}$, мм'] = self.l_0
* self.lambd
116.         
117.         def delta_t_d(p, h_0, h_1):
118.             return p * sympy.ln(h_0 / h_1).n(5) * 
0.8 / (7900 * 548) * 10 ** 6
119. 
120.         def delta_t_v(h_0, h_1, t_vh, l, v):
121.             #return 0
122.             return 4.87 / (h_0 + h_1) * (t_vh - 50
) * (2 * l * h_0 / (10 ** 3 * (h_0 + h_1) * v)) ** 0.5
123. 
124.         def delta_t_l(t_vh, t_d, t_v, h_1, l_mk, l
_1, v, v_vh_next):
125.             t = t_vh + t_d - t_v + 273
126.             tau = l_mk / v + (l_1-l_mk) / v_vh_nex
t # так как моя заготовка > 4 метров
127.             return 17.5 * t ** 4 / h_1 * tau * 10 
** (-15)
128. 
129.         def delta_t_k(t_l):
130.             return 0.03 * t_l
131. 
132.         def t_vih(t_vh, p, h_0, h_1, v, l, l_mk, l
_1, v_vh_next):
133.             t_d = delta_t_d(p, h_0, h_1)
134.             t_v = delta_t_v(h_0, h_1, t_vh, l, v)
135.             t_l = delta_t_l(t_vh, t_d, t_v, h_1, l
_mk, l_1, v, v_vh_next)
136.             t_k = delta_t_k(t_l)
137.             return t_vh + t_d - t_v - t_l - t_k, t
_d, t_v, t_l, t_k
138.         
139.         params[r'$T_{ {вых}}$, $^{\circ}C$'], 
t_d, t_v, t_l, t_k = t_vih(self.t_vh, p_mid, self.h_0, 
self.h_1, self.v, params['$l$, мм'], self.a, params[r'$
l_{ {вых}}$, мм'], self.v_vh_next)
140.         
141.         params[r'$\triangle T_{ {д}}$, $^{\cir
c}C$'] = t_d
142.         params[r'$\triangle T_{ {в}}$, $^{\cir
c}C$'] = t_v
143.         params[r'$\triangle T_{ {л}}$, $^{\cir
c}C$'] = t_l
144.         params[r'$\triangle T_{ {к}}$, $^{\cir
c}C$'] = t_k
145.         
146.         params['$\\tau$'] = self.a / (self.v * 100
0) + (params[r'$l_{ {вых}}$, мм']-self.a) / (self.v
_vh_next * 1000)
147.         
148.         self.params = params
149.         self.is_computing = True
150.     
151.     def return_params(self):
152.         if not self.is_computing:
153.             self.computing()
154.         return self.params
155. 
156. class LetStripStand(Stand):
157.     def __init__(self, f_start, f_end, b_start, b_
end, d_valkov, t_vh, a, n, lambd, l_0, v_vh_next):
158.         h_0 = f_start / b_start
159.         h_1 = f_end / b_end
160.         d_kat = d_valkov - f_end / b_end
161.         r_kat = d_kat / 2
162.         v = (3.1415 * r_kat * 2 * n) / 60_000
163.         self.v_vh = v / lambd
164.         #r_kat, h_0, h_1, b_0, t_vh, a, v, lambd, 
l_0, v_vh_next = None, b_1 = None
165.         super().__init__(r_kat, h_0, h_1, b_start,
t_vh, a, v, lambd, l_0, v_vh_next, b_end)
166.     
167.     def computing(self):
168.         super().computing()
169.         self.params['$v_{\ {вх}}$'] = self.v_v
h
170.         self.params['$v_{\ {вых}}$'] = self.v
171. 


\end{verbatim}
\end{document}